\section{Prototypage}
\subsection{Implémentation}
\begin{frame}
 \frametitle{Prototypage}
 Un réseau de capteurs sans fil a été déployé dans un maison pendant 12 jours dans un contexte de gestion d'énergie.
 L'implémentation de leur prototype ne requiert que 15,8k, parseur XML, serveur HTTP et file TCP/IP inclus !
 %TODO inclure figure 11
\end{frame}

\begin{frame}
 \frametitle{Implémentation}
 \framesubtitle{Mémoire utilisée}
 %TODO inclure table 5
\end{frame}

%TODO description de figure 11

\subsection{Déploiement}
\begin{frame}
 \frametitle{Déploiement}
 Le réseau de capteurs déployé sont aussi des capteurs communément utilisés pour la sécurité.
 Par exemple, ils ont utilisés des capteurs de mouvement, d'ouverture de porte, ...
 Mais ils ont également placé des capteurs permettant de mesurer et regler la consomation d'énergie de n'importe quel appareil grâce à leurs \textit{smart-sockets}.
\end{frame}

\begin{frame}
 \frametitle{Diagramme du \textit{smart-socket}}
 %TODO inclure figure 12
 À brancher au réseau électrique, le \textit{smart-socket} permet de mesurer l'énergie électrique consommée par l'appareil branché au \textit{smart-socket} et de l'éteindre ou l'allumer à distance.
 Il supporte jusqu'à 2kW\\
 %On a besoin de visualiser la consomation d'énergie de chaque appareil.
 %Combiné avec des capteurs d'ouverture de porte et capteur de mouvements, un algorithme de gestion d'énergie peut être appliqué.
 %Lors de l'expérience, des smart-sockets ont été utilisés pour les lampes et appareils de divertissement utilisés la plupart du temps.
\end{frame}

\begin{frame}
 \frametitle{Exemple}
 %TODO figure 13 en séparant les graphiques par des captions
 L'unité de l'axe du temps n'est pas affiché pour protéger la vie privée des résidents.
\end{frame}

\begin{frame}
 \frametitle{Résultats}
 Grâce à un algorithme qui baisse la température du chauffage lorsque la maison est innocupée, ils ont pu économiser 7,2\% d'énergie de chauffage.
 Le graphique suivant montre l'économie totale d'énergie par jour.
 %TODO inclure figure 14
 Ils ont utilisé un algorithme se basant que sur l'occupation de la maison. Il existe de meilleurs algorithmes prenant en compte l'occupation de chaque pièces.
\end{frame}

 
 