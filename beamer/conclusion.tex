\section{Conclusion}
\subsection{}

\begin{frame}
 \frametitle{Problèmes}
 \begin{itemize}
  \item \textbf{Sleep mode}: Comment s'assurer que tous les échanges de messages dans la couche réseaux soient designés pour le sleep mode ?
  \item \textbf{Multi-hop connection}: Et si le message doit passer par plusieurs capteurs pour atteindre le controleur ?\\
  Alors chaque capteur intermédiaire doit envoyer des messages wake-up, maintenir une liste d'adresse MAC...Tout cela va créer de l'overhead
  \item \textbf{Sécurité}: Comment assurer la sécurité entre un capteur et un controleur ?\\
  %Il vaut mieux éviter de crypter un paquet 6lowpan pour qu'il puisse être converti en IPv6 sans dévoir décrypter.\\
  %CC2420 crypte et décrypte en hardware en définissant quelle partie il faut crypter. Ainsi la conversion 6lowpan à IPv6 se fait sans devoir décrypter le message.\\
  %Si la clé est échangée en ligne, la proximité des deux appareils réduit les risques d'écoutes indiscrètes.
  %Ce qui n'est pas très convainquant en terme de Sécurité
  %Ou alors on transmet manuellement la clé.
 \end{itemize}
\end{frame}

\begin{frame}
 \frametitle{Résultats du prototype}
 %Grâce à un algorithme qui baisse la température du chauffage lorsque la maison est innocupée, ils ont pu économiser 7,2\% d'énergie de chauffage.\\
 Économie totale d'énergie par jour:
 \begin{figure}
  \centering
  \includegraphics[scale=0.38]{figures/energysaver.jpg}
  \caption{Économie d'énergie totale}
 \end{figure} 
\end{frame}